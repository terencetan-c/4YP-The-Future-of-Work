\documentclass[11pt]{article}
\usepackage[a4paper, margin=20mm]{geometry}
 

\usepackage{amsmath}
\usepackage{physics}

\usepackage{graphicx}
\graphicspath{ {./figs/} }
\usepackage{subfig}

%\renewcommand{\baselinestretch}{2}
\usepackage{setspace}
\doublespacing
\usepackage{titlesec}
\usepackage{authblk}
\usepackage{titling}

\setlength{\droptitle}{-9em}   % This is your set screw


\titlespacing\section{0pt}{0pt plus 0pt minus 0pt}{0pt plus 2pt minus 2pt}
\titlespacing\subsection{0pt}{0pt plus 4pt minus 2pt}{0pt plus 2pt minus 2pt}
\titlespacing\subsubsection{0pt}{0pt plus 4pt minus 2pt}{0pt plus 2pt minus 2pt}


\title{The Future of Work - 12653}
\author{Author: Nian Yang Terence Tan \and Supervisor: Dr. Michael A Osborne}

\begin{document}

\maketitle


\section{Overview of the project}
The world population is ageing over the next few decades. The rising elderly to working age population ratio is increasing and will continue to do so. This will strain the public and social services of many countries around the world. 

In this project, we aim to examine the relationship between the age distribution within occupations and the degree of automation (both present and future) of those occupations. We might also look into any correlations with the skills/knowledge required for those occupations. This will all be done using a machine learning technique called Gaussian Process.

\section{Key project objectives}
\label{sec:RBF}

% \begin{figure}[!htb]
%     \centering
%     \subfloat[Without White Kernel (RMS error = 1.748)]{\includegraphics[width=8cm]{Figures/Graph 1.png}\label{fig:fig1}}
%       \hfill
%     \subfloat[With White Kernel (RMS error = 0.667)]{\includegraphics[width=8cm]{Figures/Graph 2.png}\label{fig:fig2}}
%     \hfill
%     \caption{Predicted values}
%   \end{figure}
  
  \section{Progress to date}
%   The data set exhibits periodic behaviour, with a period of roughly 50000 seconds. Hence, we attempt to use a kernel which can model periodicity. 
  
%   Figure \ref{fig:fig3} is similar to Figure \ref{fig:fig2} despite not including a White Kernel. However, the former retains the inaccurate predictions at the same points but has a lower RMS error. On the other hand, Figure \ref{fig:fig4} (RMS error of 0.192) gives the best predictions without the same errors present in Figures \ref{fig:fig2} and \ref{fig:fig3} (RMS error of 0.433). Again, including a White Kernel improved performance. These results seem to validate our assumptions that a periodic kernel would be a better model of the data.


  
%   \begin{figure}[!htb]
%       \centering
%       \subfloat[Without White Kernel (RMS error = 0.433)]{\includegraphics[width=8cm]{Figures/Graph 3.png}\label{fig:fig3}}
%         \hfill
%       \subfloat[With White Kernel (RMS error = 0.192)]{\includegraphics[width=8cm]{Figures/Graph 4.png}\label{fig:fig4}}
%       \hfill
%       \caption{Predicted values}
%     \end{figure}
    
  \section{Immediate tasks}

  \section{Plan of work to the end of the project}

\end{document}
